\documentclass{article}

\usepackage{amsmath, amssymb, amsthm}
\usepackage[margin=1in]{geometry}
\usepackage{tcolorbox}
\usepackage{fancyvrb}
\usepackage[hidelinks]{hyperref}
\usepackage{tikz, wasysym, tikz-cd}
\usepackage{mathtools} %xrightarrow
\usetikzlibrary{automata,arrows.meta,positioning,calc,decorations.pathmorphing}
\usepackage{etoolbox} % setcounter{counter}{num}
\usepackage{enumitem} % \begin{enumerate}[label=(\alph*)]
\usepackage{xcolor}

% QUIVER STUFF
% A TikZ style for curved arrows of a fixed height, due to AndréC.
\tikzset{curve/.style={settings={#1},to path={(\tikztostart)
    .. controls ($(\tikztostart)!\pv{pos}!(\tikztotarget)!\pv{height}!270:(\tikztotarget)$)
    and ($(\tikztostart)!1-\pv{pos}!(\tikztotarget)!\pv{height}!270:(\tikztotarget)$)
    .. (\tikztotarget)\tikztonodes}},
    settings/.code={\tikzset{quiver/.cd,#1}
        \def\pv##1{\pgfkeysvalueof{/tikz/quiver/##1}}},
    quiver/.cd,pos/.initial=0.35,height/.initial=0}
% TikZ arrowhead/tail styles.
\tikzset{tail reversed/.code={\pgfsetarrowsstart{tikzcd to}}}
\tikzset{2tail/.code={\pgfsetarrowsstart{Implies[reversed]}}}
\tikzset{2tail reversed/.code={\pgfsetarrowsstart{Implies}}}
% TikZ arrow styles.
\tikzset{no body/.style={/tikz/dash pattern=on 0 off 1mm}}

\theoremstyle{plain}
\newtheorem{theorem}{Theorem}
\numberwithin{theorem}{section}
\newtheorem{lemma}{Lemma}
%\numberwithin{lemma}{section}
\newtheorem{example}{Example}
\numberwithin{example}{section}

\theoremstyle{definition}
\newtheorem{definition}{Definition}
\numberwithin{definition}{section}
\newtheorem{problem}{Problem}
%\numberwithin{problem}{section}

\newenvironment{solution}
  {\renewcommand\qedsymbol{$\blacksquare$}\begin{proof}[Solution]}
  {\end{proof}}

\newtcolorbox{codebox}[1][] {
    colback = black!2!white,
    colframe = black,
    sharp corners,
    title = \texttt{#1}
}

\newtcolorbox{problembox} {
    colback = white,
    colframe = black,
    sharp corners
}

\newtcolorbox{defbox} {
    colback = blue!5!white,
    colframe = blue!75!black,
    sharp corners
}

\newtcolorbox{lemmabox} {
    colback = green!5!white,
    colframe = green!75!black,
    sharp corners
}

\newtcolorbox{theorembox} {
    colback = red!5!white,
    colframe = red!75!black,
    sharp corners
}

\newcommand{\Sup}{\text{Sup}}
\newcommand{\Inf}{\text{Inf}}
\DeclareMathOperator{\im}{im}
\DeclareMathOperator{\dom}{dom}
\DeclareMathOperator{\cod}{cod}
\DeclareMathOperator{\id}{id}
\newcommand{\angles}[1]{\left<#1\right>}
\newcommand{\catname}[1]{{\normalfont\textbf{#1}}}
\newcommand{\GL}{\text{GL}}
\newcommand{\SL}{\text{SL}}
\newcommand{\sg}{\text{sg}}
\newcommand{\dd}{\mathop{}\!d}
\newcommand\defeq{\stackrel{\mathclap{\normalfont\mbox{\scriptsize def}}}{=}}

%%%%%%%%%%%%%%%%%%%%%%%%%%%%%%%%%%%%%%%%%%%%%%%%%%%%%%
% \set, my favourite
%%%%%%%%%%%%%%%%%%%%%%%%%%%%%%%%%%%%%%%%%%%%%%%%%%%%%%
\DeclarePairedDelimiterX\set[1]\lbrace\rbrace{\def\given{\;\delimsize\vert\;}#1}
 
\newcommand{\naturalto}{%
  \mathrel{\vbox{\offinterlineskip
    \mathsurround=0pt
    \ialign{\hfil##\hfil\cr
      \normalfont\scalebox{1.2}{.}\cr
%      \noalign{\kern-.05ex}
      $\longrightarrow$\cr}
  }}%
}

% Actions to be done at the start of a new section, so just made a new macro
\newcommand{\Section}[1]{\section{#1} \setcounter{equation}{0}}

\setlength{\parindent}{0pt}

\title{Project 1 - 2023}
\author{Mitchell Holt (46977557)}

\begin{document}

\maketitle

\tableofcontents

\newpage

\section{GitHub Repository}

The entire source code for the project can be found on the
\href{https://github.com/mitchellholt/Mitchell-Holt-2504-2023-PROJECT1}
{\color{blue}GitHub repository}.


\section{Example Script}
The provided example script demonstrates some of the functionality of the
pre-existing code. The first few lines initialize the project, and then there is
an example of some basic polynomial arithmetic:

\begin{codebox}
    \begin{Verbatim}
x = x_poly(Polynomial)
p1 = 2x^3 + 4x^2 - 3x
p2 = 2x^4 - 4x^2 - 3x + 3

@show p1*p2
@show p1^3
    \end{Verbatim}
\end{codebox}

Output:
\begin{Verbatim}
    p1 * p2 = 4x^7 + 8x^6 - 14x^5 - 22x^4 + 6x^3 + 21x^2 - 9x
    p1 ^ 3 = 8x^9 + 48x^8 + 60x^7 - 80x^6 - 90x^5 + 108x^4 - 27x^3
\end{Verbatim}
We may also take the derivatives of polynomials:

\begin{codebox}
    \begin{Verbatim}
@show derivative(p1*p2)
@show derivative(p1)*p2 + p1*derivative(p2);
    \end{Verbatim}
\end{codebox}

Output:
\begin{Verbatim}
    derivative(p1 * p2) = 28x^6 + 48x^5 - 70x^4 - 88x^3 + 18x^2 + 42x - 9
    derivative(p1) * p2 + p1 * derivative(p2) = 28x^6 + 48x^5 - 70x^4 - 88x^3 + 18x^2 + 42x - 9
\end{Verbatim}

There are functions defined for working with polynomials modulo some prime,
which is particularly useful for factoring polynomials:

\begin{codebox}
    \begin{Verbatim}
prime = 17
p = mod((7x^3 + 2x^2 + 8x + 1)*(x^2+x+1),prime)
println("Will factor this polynomial (mod $prime): ", p)
factorization = factor(p,prime)
println("Here is the factorization: ", factorization)

pr = mod(expand_factorization(factorization),prime)
println("Reconstructing: ", pr)
    \end{Verbatim}
\end{codebox}

Output (with the third line split so the output will fit on the document):
\begin{Verbatim}
    Will factor this polynomial (mod 17): 7x^5 + 9x^4 + 11x^2 + 9x + 1
    Here is the factorization: Tuple{Polynomial, Int64}[(x + 8, 1),
        (x^2 + x + 1, 1), (x^2 + 2x + 7, 1), (7, 1)]
    Reconstructing: 7x^5 + 9x^4 + 11x^2 + 9x + 1
\end{Verbatim}

\bigbreak

\section{Tests}
The provided tests produce the following output:
\begin{Verbatim}
    test_euclid_ints - PASSED
    test_ext_euclid_ints - PASSED
    prod_test_poly - PASSED
    prod_derivative_test_poly - PASSED
    ext_euclid_test_poly - PASSED
    division_test_poly - PASSED

    doing prime = 5 	 ..........
    doing prime = 17 	..........
    doing prime = 19 	..........
    factor_test_poly - PASSED
\end{Verbatim}

\bigbreak

\section{Task 1}
\subsection{Example Script}
TODO lol

\bigbreak

\subsection{Pretty Printing}
The show methods in both \texttt{polynomial.jl} and \texttt{term.jl} were
modified to meet the pretty printing rules. A global option was added to
determine if terms in polynomials were printed in ascending or descending term
order (defaulting to descending):

\begin{codebox}[polynomial.jl]
    \begin{Verbatim}[numbers=left,xleftmargin=5mm]
"""
Based global variable for deciding the order in which polynomial terms are
printed
"""
global lowest_to_highest :: Bool = false

    \end{Verbatim}
\end{codebox}

The provided \texttt{show} method for polynomials was replaced with the
following:

\begin{codebox}[polynomial.jl]
    \begin{Verbatim}[numbers=left,xleftmargin=5mm]
function show(io::IO, p::Polynomial) 
    if iszero(p)
        print(io,"0")
    else
        is_first = true
        n = length(p.terms)
        term_list = lowest_to_highest ? p.terms : reverse(p.terms)
        for t in term_list
            if iszero(t)
                continue
            elseif is_first
                print(io, t.coeff < 0 ? "-" : "", t)
            else
                print(io, t.coeff < 0 ? " - " : " + ", t)
            end
            is_first = false
        end
    end
end
    \end{Verbatim}
\end{codebox}

First, observe that the \texttt{lowest\_to\_highest} flag is used on line 7 to
determine in which order to iterate over the list of terms. If the given
polynomial is the zero polynomial, then the output is \texttt{0}. Otherwise, the
\texttt{show} method above will simply skip those terms which are zero.
Printing terms with negative coefficient as \texttt{1 - x} instead of
\texttt{1 + -x} was implemented by changing the monomial show method to always
display the absolute value of the coefficient. Then on line 14 above, the
\texttt{show} method prints either a \texttt{+} or \texttt{-} sign in front of
the monomial as appropriate. \medbreak

The final consideration is how to print the first term in the monomial so that
there is not ``\texttt{ + }" or ``\texttt{ - }" with excessive surrounding
whitespace at the start of the \texttt{show} output. It is not necessarily the
case that the first term in the list of terms for the polynomial will be
non-zero, so we introduce on line 5 a flag \texttt{is\_first}, which will be set
until the first non-zero term has been printed. Then at the first term we need
only check if it has positive coefficient (and print nothing before the term),
or negative (in which case we print ``\texttt{-}" with no surrounding
whitespace). \medbreak

Consider now the implementation of the \texttt{show} method for monomials:

\begin{codebox}[term.jl]
    \begin{Verbatim}[numbers=left,xleftmargin=5mm]
function show(io::IO, t::Term)
    if t.degree == 0
        print(io, abs(t.coeff))
    elseif abs(t.coeff) == 1
        print(io, "x", t.degree == 1 ? "" : int_to_superscript(t.degree))
    else
        print(io, abs(t.coeff), "x",
            t.degree == 1 ? "" : int_to_superscript(t.degree))
    end
end
    \end{Verbatim}
\end{codebox}

We check for zero degree terms (and print the coefficient only) on lines 2-3. At
line 4-5 we check if a polynomial has non-zero degree and has a coefficient of
$\pm 1$, in which case we print \texttt{x} instead of \texttt{1x}. Furthermore,
on lines 5 and 7-8, we only print the degree of the monomial if it not 1.
\medbreak

Finally, unicode superscript printing was implemented using a function
\texttt{int\_to\_superscript}:

\begin{codebox}[general\_alg.jl]
    \begin{Verbatim}[numbers=left,xleftmargin=5mm]
function digit_to_superscript(n :: Int) :: Char
    # Literal for zero
    zero_code = 0x2070
    
    # cringe edge cases
    if n == 1
        return Char(0x00B9)
    elseif n == 2
        return Char(0x00B2)
    elseif n == 3
        return Char(0x00B3)
    else
        # based normal case
        return Char(zero_code + n)
    end
end

function int_to_superscript(k :: Int) :: String
    k == 0 ? "" : int_to_superscript(div(k, 10)) * digit_to_superscript(mod(k, 10))
end
    \end{Verbatim}
\end{codebox}

The function \texttt{digit\_to\_superscript} converts a single digit to a
superscript unicode character and is called in the recursively defined
\texttt{int\_to\_superscript} on line 18. Notice that the \texttt{*} operator on
line 19 is the string concatenation operator. Although converting the input
\texttt{k} into a string and then mapping each character to a superscript one
would have worked, this method is much more efficient as no memory is allocated
for an intermediate string representation of the input.
\end{document}
